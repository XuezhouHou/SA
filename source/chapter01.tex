% !TeX spellcheck = en_US
\glsresetall%
\chapter{Introduction}%
This article highlights the importance of analyzing communication 
delays in a modular way using two industry use cases for \gls{mas}.
Before diving into the actual work of the research, 
a brief motivation followed by the research questions will be
provided to guide this study.

\section{Motivation}\label{chap: Motivation}
The requirements of modern industrial production have become more and 
more crucial due to the increased complexity of interconnections of 
various robots and automation systems during manufacturing. Different 
from the production mode of a traditional factory, the concept of a 
smart factory in Industry 4.0 has been developed to overcome the 
remaining issues, such as the high centralization of a traditional 
control system, low scalability of production systems, and processes, 
limited adaptability of new conditions and requirements with changing 
environments, hardness of real-time decision making, insufficient resource 
allocation and many more. The debate about centralized and decentralized 
production planning already lasts long\cite{eloranta2014designing}. 
Both of them have advantages and disadvantages against 
each other. The choice of one over another mostly depends on the 
production demands. A centralized system mainly controls lower-level system components, including  
direct operations on middle and lower-level components in a hierarchical 
pattern. On the other hand, each component of a decentralized system 
is equally 
important for contributing to the global goal known as distributed 
control. Whether to choose a centralized or decentralized production system 
remains a tradeoff between different production purposes. 
Under this prerequisite, a \gls{mas} offers a possible solution for the 
questions. 


The precursor to modern \gls{mas} is \gls{dai}, which focuses on distributing a 
single complex AI task to multiple machines and processors\cite{noauthor_jacques_nodate}, 
which is nowadays classified into parallel AI, \gls{dps}, and \gls{mas}\cite{dorri_multi-agent_2018}. 
With the focus on \gls{mas} in this article, it can be described as a combination of 
autonomous agents and their environments. However, the complete decomposition of 
the complex real-world system can be challenging, which leaves space for self-learning 
and decision-making in a primitive level of agent autonomy\cite{reis_applications_2004}. 
The concept of resource distribution/decentralization is inherited by \gls{mas} to 
handle more distributed and interconnected computing tasks. It results in a more 
intelligent and autonomous agent-based operation system in which each agent can make 
decisions independently to achieve its own goal. At the same time, they still tend 
to collaborate, negotiate, and coordinate with other agents frequently in order to 
improve the efficiency and quality of production workflow meanwhile reducing 
cost \cite{vogel-heuser_multi-agent_2020}. In a \gls{mas}, agents have the ability 
to interact with one another, perceive their surroundings, and adapt to rapid changes 
in the environment. This collective capability enables them to solve complex problems 
that a single agent would not be able to handle alone. 


Effective communication between agents is crucial for the successful functioning of a 
system\cite{georgeff_communication_1988}. According to the discussion of agent 
communication\cite{vogel-heuser_multi-agent_2020}, in addition to the horizontal 
communication between agents, the vertical communication between assets 
and their mapped \gls{aas} can also be realized by agents to extend the domain 
of production lifecycle. \gls{aas} is a standard 
that serves as the foundation of the \gls{dt} implementation\cite{redeker_towards_2021}. 
To differentiate, \gls{aas} provides a data structure for \gls{dt}, while Digital 
Twin uses the data for real-time data representation, simulation, and analysis. 
The concept of \gls{dt} was first introduced by Michael 
Grieves \cite{flumerfelt_complex_2019}, who also introduce the 
famous \gls{plm} model \cite{greengard_digital_nodate}to 
explain the role of a in the product lifecycle. Data from engineering, 
design and manufacture should be digitalized to represent physical 
assets. Some common understandings of engineering data could be 
for example simulation for performance testing, design data 
that builds a visual representation of CAD data of plants and robots, 
and manufacture data that helps to inspect changes of production for 
process optimization. 
Based on a summation of timeline-based \gls{dt} definitions until the year 
2016\cite{negri_review_2017}, the concept of \gls{dt} is defined as: \textit{"A 
unified system model that can coordinate architecture, mechanical, electrical, 
software, verification, and other discipline-specific models across the system 
lifecycle, federating models in multiple vendor tools and configuration-controlled 
repositories"}\cite{bajaj_architecture_2016}. Therefore, the research on agent-based 
communication between physical assets and \gls{dt} will be another focus of this 
article. 


When exploring \gls{dt} and \gls{mas}, a network system is essential for agent communication 
and therefore the demands of resolving network delays during data exchange is growing. The network 
latency refers to the time it takes for a packet to travel across the network, from one end 
to the other end, namely the netwotk communication delay. The network delay can be measured 
as either \gls{rtt}, representing the time consumption of a packet travels from source and 
back, or \gls{owd}, the one-way latency from source to 
destination. One solution for \gls{owd} estimation\cite{abdou_accurate_2015} is 
to halve the calculated \gls{rtt}\cite{karn_improving_nodate}, with the \gls{owd} 
value synchronized before meausurement\cite{abdou_accurate_2015}. 


Not only the network delays influence the system performance, timing behaviors of other 
entities for example the system control delays, should also be considered for the \gls{mas} and 
\gls{dt} system.
Just as a factory's production shifts from standalone operations to a distributed, 
agent-based system, there is a parallel need to evolve the representation of timing 
properties. Moving from a generic approach, it becomes essential to adopt a more 
specific model, such as using a \gls{dsl} to ensure the resolution of problems 
in the level of abstraction of the problem domain. The definition of \gls{dsl} model 
can be formulated as follows: \textit{"\gls{dsl} is a programming language or executable 
specification language
that offers, through appropriate notations and abstractions, expressive power focused 
on, and usually restricted to, a particular problem domain."}\cite{van_deursen_domain-specific_2000}
Compared with \gls{gpl}, \gls{dsl} tends to solve a problem in a more efficient 
and optimitzed way within a specific domain, making it easier for domain experts 
with limited programming experiences to work with it.  

With the development of automation in industrial production, more studies 
should be done to move the production processes from a traditional, highly 
centralized mode to a decentralized, interconnected pattern, along with 
a higher system modularization possibility. 









\section{Research questions}
\textbf{RQ1}: \textit{How to decentralize a general \gls{mas} from a traditional highly 
centralized pattern to obtain a more efficient and real-time oriented agent-based 
communication system?}

In a collaborative robot system, planning and scheduling are always a topic under 
discussion. In a centralized \gls{mas}, the management system always has full control 
of the whole process plans, which allows the designer to have a full understanding 
of the process. However, the collaboration will highly rely on that one control 
unit, resulting in a full collapse of the system with faults. One the other hand, 
the decentralization of \gls{mas} is often relies on local decision-making from 
a global state, to achieve a local goal. In comparison, it becomes more difficult for an agent to 
coordinate and synchronize with the others, causing system inconsistency and 
efficiency reduction\cite{xuan_multi-agent_nodate}\cite{sharma_comparative_2016}.




\textbf{RQ2}: \textit{How to measure the network delays of data exchanges between 
different agents in \gls{mas}, as well as delays between local devices and the cloud 
to retrieve end-to-end delays for each participant of the production system?}

Delay measurement is always a key factor in checking the real-time capabilities of a 
production system. The most common delay measurement of the network is based on 
\gls{rtt} and \gls{owd}, meaning the cycle time and one-way time it takes for 
a message to be transferred 
from the source to the destination port\cite{paxson_end--end_1999}. The value of the measured delays can be influenced by 
many factors, including network bandwidth and throughput, number of users, hardware 
or cables as physical mediums, and many more\cite{neokosmidis_roadmap_2017}. 



\textbf{RQ3}: \textit{How to design a communication-based \gls{mas} for two industrial use 
cases that is also adaptable for other use cases with different production 
requirements?}
Due to the complexity of \gls{mas} design, we mainly focus on agent-based communication, 
planning, decision-making, and scheduling. The goal is to design a \gls{mas} architecture, 
that can be adapted to any production requirements. The designed architecture should 
be later tested and verified by two industrial use cases BMW and Wittenstein. 

\textbf{RQ4}: \textit{How to modularize the delays among the \gls{mas} to improve the 
predictability, scalability, accuracy, and troubleshooting possibilities for 
each module?}  

One challenge is to analyze the network delays in \gls{cps}. Whether the delays fulfill 
the real-time requirements or not is less determined\cite{forouzan2004data}. Because the real-time requirements 
are changing over factors such as product types, customer requirements, resources, hardware 
capabilities, and so forth. So it is impossible to analyze the delays directly with 
the data. One possible solution is to modularize the delays to have a better understanding 
of the timing properties in each domain of the module\cite{blecker_modularity_2006}. 




\section{Outline}
In this article, chapter \ref{chap: S-O-T-A} talks about the state of the art and 
the unsolved issues in the current study of \gls{mas}, network 
latency, current approaches of \gls{dsl} 
based graphical modelings for \gls{cps}, the \gls{dt} with 
its application in different fields and the real-time requirements 
for robot-based production system. Chapter \ref{chap: Preliminary} 
provides the prelimilary for the sytem design. After that, 
chapter \ref{chap: Meth} explains 
the methodologies applied to \gls{mas} and \gls{dta} programming, 
followed by an assumption of modularizing \gls{mas} especially the agent-based 
communication system. By utilizing the methods, programs are 
designed for \gls{mas} and \gls{dta} based on two industrial use cases, 
with their performances tested under different conditions. 
Further more, test cases and one use case are 
performed in chapter \ref{chap: Result}, along with 
the discussions for the results. Finally, the last chapter summarizes 
all the general \gls{mas} related research in this article to make a 
conclusion and to provide insights into future research directions.  