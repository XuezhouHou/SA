% !TeX spellcheck = en_US
\glsresetall%
\chapter{Introduction}%
This article is written to raise readers' interest in the necessity of 
analyzing communication delays between each participant in a general 
\gls{mas} based on two industry use cases in a more modular way. 
Before getting deep into the actual work of the research, 
a brief motivation followed by the research questions will be
provided to guide this study.

\section{Motivation}
The requirements of modern industrial production has become more and more crucial, due to the increased complexity of interconnections of various different robots and automation systems during manufacturing. Different from the production mode of traditional factory, the concept of smart factory of Industry 4.0 has been developed to overcome the remaining issues such as high centralization of traditional control system, low scalability of production systems and processes, limited adaptability of new conditions and requirements with changing environments, hardness of real-time decision making, insufficient resource allocation and many more. Under this prerequisite, a \gls{mas} plays a crucial part to close the gaps. 


The precursor to modern \gls{mas} is \gls{dai}, which focuses on distributing a single complex AI task to multiple machines and processors\cite{noauthor_jacques_nodate}. The concept of resource distribution/decentralization is inherited by \gls{mas} to handle more distributed and interconnected computing tasks and results in a more intelligent and autonomous agent based operation system, in which each agent can make decision independently to achieve its own goal while still tend to collaborate, negotiate and coordinate with other agents frequently, in order to improve the efficiency and quality of production workflow meanwhile reducing cost \cite{vogel-heuser_multi-agent_2020}. The capability of agents being able to interact with each other, perceive and adapt to the rapid changes in the environment makes it possible for \gls{mas} to solve comprehensive problems which a single agent cannot. 





\section{Research questions}
RQ1: How to decentralize a general \gls{mas} from a traditional highly 
centralized pattern to obtain a more efficient and real-time oriented agent-based 
communication system?

RQ2: How to measure the network delays of data exchanges between 
different agents in \gls{mas}, as well as delays between \gls{dta} and the cloud 
to retrieve end-to-end delays for each participant of the production system?

RQ3: How to modularize the delays among the general \gls{mas} to improve the 
predictability, scalability, accuracy, and troubleshooting possibilities for 
each module?  



\section{Outline}
In this article, chapter 2 talks about the state of the art and 
the unsolved issues in the current study of \gls{mas}, network 
latency and how it is measured, current approaches of \gls{dsl} 
based graphical modelings for \gls{cps}, and the Digital Twin with 
its application in different fields. After that, chapter 3 explains 
the methodologies applied to \gls{mas} and \gls{dta} programming, 
followed by an assumption of modularizing network latency as well 
as robot control delays. By utilizing the methods, programs are 
designed for \gls{mas} and \gls{dta} based on two industrial use cases, 
and tests for different. Further more, test cases and one use case are 
performed in chapter 4, along with 
the discussions for the results. Finally, the last chapter summarizes 
all the general \gls{mas} related research in this article to make a 
conclusion and to provide insights into future research directions.  