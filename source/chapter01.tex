% !TeX spellcheck = en_US
\glsresetall%
\chapter{Introduction}%
%\gls{am} is the umbrella term for a variety of technologies, describing the automated process of manufacturing physical parts directly from virtual \gls{cad} models. %
%\gls{pbflbp}, also known as \gls{sls}, as part of the \gls{am} process group
\section{Motivation}
The requirements of modern industrial production has become more and more crucial, due to the increased complexity of interconnections of various different robots and automation systems during manufacturing. Different from the production mode of traditional factory, the concept of smart factory of Industry 4.0 has been developed to overcome the remaining issues such as high centralization of traditional control system, low scalability of production systems and processes, limited adaptability of new conditions and requirements with changing environments, hardness of real-time decision making, insufficient resource allocation and many more. Under this prerequisite, a \gls{mas} plays a crucial part to close the gaps. 


The precursor to modern \gls{mas} is \gls{dai}, which focuses on distributing a single complex AI task to multiple machines and processors\cite{noauthor_jacques_nodate}. The concept of resource distribution/decentralization is inherited by \gls{mas} to handle more distributed and interconnected computing tasks and results in a more intelligent and autonomous agent based operation system, in which each agent can make decision independently to achieve its own goal while still tend to collaborate, negotiate and coordinate with other agents frequently, in order to improve the efficiency and quality of production workflow meanwhile reducing cost \cite{vogel-heuser_multi-agent_2020}. The capability of agents being able to interact with each other, perceive and adapt to the rapid changes in the environment makes it possible for \gls{mas} to solve comprehensive problems which a single agent cannot. 





\section{Research questions}
RQ1: How can agents communicate with each other in \gls{mas}?


RQ2: How to measure the delays of data exchanges between different \gls{ra} and delays of data upload/download within \gls{dta}?

RQ3: How to modularize the delays with \gls{dsl}?



\section{Outline}
In this thesis, chapter 2 introduce the concepts and utilization of MAS, the Network communication principles and protocols, \gls{dsl} for \gls{cps} with the concentration of robotics, anda. Chapter 3 summarizes the state of art of all those concepts in chapter 2. After that, chapter 4 comes up with methodologies of building a \gls{mas} for communication and aa agent for data update and chapter 5 comes up with the implementation results of different test cases and two use cases for performance testing on the \gls{mas} anda agent.