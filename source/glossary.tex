\newglossaryentry{latex}%
{%
	name=Latex,%
	description={Generische Mark-up-Sprache zum Erstellen wissenschaftlicher Texte. Sie ist in jeder Hinsicht Word überlegen, welches einem visuellen Mark-Up entspricht. Das X von \LaTeX\ wird als ç (Stimmloser palataler Frikativ) ausgesprochen, vergleiche deutsche Aussprache von \textit{ch}.}%
}%
\newglossaryentry{tutorial}%
{%
	name=Tutorial,%
	description={Kurze Gebrauchsanleitung welche ein Thema, einen gewissen Vorgang oder eine Funktion erklärt. Hat nicht den Anspruch auf Vollständigkeit.}%
}%
\newglossaryentry{tikz}%
{%
	name=Ti\textit{k}Z,%
	description={Frontend-Paket, das auf PGF-Plot aufbaut und zum Erstellen von Graphiken dient.}%
}%