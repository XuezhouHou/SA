\chapter{Conclusion and outlook}

In this article, we emphasize that in modern factories, in order to handle 
multi-tasking problems, it is essential to introduce the concept of system 
decentralization. Considered a modern and effective solution to system 
decentralization, \gls{mas} is incredibly functional for domain-specific 
applications. By applying \gls{mas} to a traditional master-slave\cite{egger_deployment-friendly_2020} 
pattern, we now obtained a system that is mainly scheduled and coordinated 
by \gls{cda} who receives the customer requirements and breaks those down to agent 
executable primitives, supplemented by other intelligent \gls{ras} that proceed 
with those primitives, to enable the 
autonomous decision-making and planning in each agent. Distinct from a 
general \gls{mas}, we extended its domain to \gls{iot} by 
introducing an additional \gls{dta} to bridge the gap between physical 
entities and \gls{dt}. On top of the existing Azure \gls{dt} architecture, we 
added an Azure Event Hub and Azure Data Explorer inside the \gls{dta} data 
flow cycle to store and analyze the data. 


To represent the systems, we designed two programs for both local 
\gls{mas} and \gls{dta}, with the name Websocket\_MAS and DTAgent 
(along with a \gls{rcp}). Although both can transfer data, the 
specific content required for production processes differs. For 
example, local \gls{mas} transmits commands for the agent's decision-making 
and planning, and \gls{dta} is responsible for the robot's raw data processing 
and routing between local devices and global \gls{dt}. Due to its high complexity 
of communication mechanism, local \gls{mas} is designed based on the application 
layer protocol WebSocket, which is exceptionally suitable for real-time required, 
bi-directional, and full-duplex agent-based communications. 
In comparison, the data transport in \gls{dta} is one-directional, which allows 
a simple, more-to-one server-client communication based on the transport layer 
protocol \gls{tcp}. The choice of \gls{tcp} minimizes the packet header and 
guarantees a stable and secure data transport through the internet. In fact, 
all the agents are connected by a 5G wireless network, which makes it possible 
to capture, compare, and analyze the delays between each other. As we all know, 
many factors influence network delays: the transmission medium (e.g., wireless), 
the distance between two transport entities, network traffic, bandwidth, 
throughput, and so forth. To ensure that the collaborative robot operation 
system meets real-time requirements, it is crucial to minimize network delays. 



In this article, we conducted many tests to analyze the network delays 
for both systems. Those tests for \gls{mas} communications can be roughly 
divided into three types: performance tests for an increasing number of servers, 
clients, and message lengths in different types, including worst-case scenarios, 
message prioritization mechanism in server design for critical messages 
(e.g., emergent stop, important task prioritization), and a use case specific 
delay measurement to simulate real-world production process. Respecting \gls{dta}, 
the tests are more straightforward. We first measured the transmission delays of 
different robot states from \gls{rcp} to \gls{dta}, and then we performed tests 
for data upload and download between \gls{dta} and Azure \gls{dt}. Based on the 
results from \cite{cainelli_performance_2023} for uplink and downlink delay measurement under a 5G 
network, this results in a more significant delay by the former and less by the 
latter. In our test, we managed to separate the data upload and download \gls{owd} 
delays from the total cycle time by extracting the twin update time in 
Azure \gls{dt}. Our results show an opposite upload and download \gls{owd}, 
possibly due to a different network or cloud setup. Since many influencing 
factors exist, we can hardly analyze the network delays simply from the values. 



Therefore, we inherited the model-driven approach from the automation industry. 
There are already extended graphical notations based on \gls{dsl} that exist 
for analyzing the timing properties for both software and hardware under soft 
and hard real-time requirements\cite{hujo_toward_2022} and the modularization 
approaches for robot-like systems\cite{volpert_supporting_nodate}. Based on these 
considerations, we designed a graphical model to describe the timing properties 
and requirements for agent-based communication and robot control systems. 
In our model, the data follows a path to flow through the \gls{tcp/ip} layers 
after being processed in \gls{cpu} and again the same layers in the opposite 
direction to the receiver \gls{cpu} endpoint. Similarly, the robot control system 
also produces delays, which can be modularized as delays in technical processing, 
I/O module, and robot controller interface. 
Another modularization consideration is related to agent-based design architecture. 
According to the Wannagat's design pattern\cite{wannagat_agent_nodate}\cite{wannagat_entwicklung_2010}, instead of the delays, the 
functionalities for each agent can be categorized to five modules:  Planning 
module, control module, diagnosis module, knowledge base, and communication 
interface. In our \gls{mas} design, each module is interconnected to transfer 
states within an agent, where a series of tasks from each module is performed.


Our graphical notations provided insight into a more abstract and domain-specific 
modularization of timing properties for \gls{mas}, and decision-making and planning 
within each agent. 
Future work could be using the existing tools to measure or simulate delays for 
each module or developing a modeling tool for research purposes. And also, one improvement 
of the \gls{dta} system design could be introduing a real-time visualization platform 
for a more intuitive representation of the physical changes in robots' collaboration.