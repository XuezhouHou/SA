% !TeX spellcheck = en_US
\chapter{State of the art}%
This chapter presents the current research on various topics discussed 
in this article, including \gls{mas}, Digital Twin, network latency, 
and \gls{dsl} as a system modularization method. The article aims to 
design a program to fulfill the real-time communication requirements 
of specific use cases, in addition to studying the timing behaviors. 
It builds on the works of leading researchers in the field.

\section{MAS}

Within the subfields of \gls{mas}, an agent can be identified as a software agent that has no physical embodiment but only software to control physical assets for different purposes. "In agent-oriented software development, an agent is the concept of a delineable software unit with a defined goal. An agent tries to achieve this goal through autonomous behavior, continuously interacting with its environment and other agents." \cite{wagner_agentenunterstutztes_2008} However, although not under discussion in this thesis, other interpretation of \gls{mas} can be a Multi-agent robot systems that each agent represents an actual physical objects such as an individual robot that participates in the complex task execution. \cite{ota_multi-agent_2006}   
Different researches define \gls{mas} from different aspects, which can be confusing for problems clarity. To simplify the concept, a general \gls{mas} can be subdivided into three views: the technical system that comprises the robots, the automation control system that is characterized by sensors, actuators, networks and robot control units, and the technical process that describes the product's production process.\cite{lauber_prozessautomatisierung_1999} \cite{wannagat_agent_nodate}The focus of this thesis is mainly in technical system which can be interpreted as a union of robots in a smart factory, and the components of a robot or functions executed for a specific movement is less under concern here. 


For further discretization of an agent, whether the agent is product, process or resource oriented, an appropriate agent architecture should be chosen according to different considerations. There are several of them should be emphasized: \gls{ra}, \gls{ca}, and \gls{ams}. \gls{ra} is agent in field level representing a single robot. Different from then other agents, \gls{ra} should be able to combine the modules with physical entities by choosing an appropriate design pattern. Therefore, a comparison of design patterns in different production levels is done \cite{ocker_leveraging_2021}. The choice of an ideal design pattern should be limited for \gls{ra} in this research by comparing three relevant design patterns:  \gls{ra} pattern in Wannagat’s architecture, \gls{mfs} patterns in Fischer’s architecture self*-control MAS in Ryashentseva’s architecture. Among all,  Wannagat’s architecture is chosen as the appropriate design patterns for \gls{ra} for field level control, which consists of five modules: Planning Module, Knowledge Base, Control Module, Diagnosis Module and communication interface. All modules are interconnected meanwhile with each connected to I/Os of physical system and a communication interface to interact with other \gls{ras} or \gls{ams} through \gls{ca} \cite{cruz_salazar_cyber-physical_2019}. \gls{ams} and \gls{ca} on the other hand should have different specifications in the same design pattern. \gls{ca} for example, should be able to coordinate the message-based communication between the agents, as a "mailbox" between them while \gls{ams} plays an important role in centralization and coordination of all other agents \cite{wannagat_entwicklung_2010}. 

\section{Network communication}
\section{DSL}
\section{Digital Twin}


Apart from the agents that are responsible for internal production 
processes, \gls{dta} plays a crucial part in collecting data from the 
production and storing it externally to create a digital replica of the 
physical entities or systems. The concept of digital twin was first introduced by Michael Grieves \cite{flumerfelt_complex_2019}, who also introduce the famous \gls{plm} conceptual diagram \cite{greengard_digital_nodate}to explain the role of a in the product lifecycle. Data from engineering, design and manufacture should be digitalized to represent physical assets. Some common understandings of a of engineering data could be for example simulation for performance testing. Or may be design data that builds a visual representation of CAD data of plants and robots, and manufacture data that helps to inspect changes of production for process optimization. However, none of these combines all data from \gls{plm} model for a digital overview of the whole factory. They can be described as local a, which should be extended to the concept of global a, by summarizing the data in a cloud platform and visualizing it with a real-time 3D graphics collaboration platform. 



With all these different types of agents following the same workflow, it is fascinating to investigate how data exchanges can be realized in a real-time behavior. Which is the main topic throughout the whole thesis.


\section{Research gap}