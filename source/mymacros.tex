% Einheitliche Schreibweise
% ---------------------------------------
\newcommand{\zb}{z.\,B.\xspace}%
\renewcommand{\dh}{d.\,h.\xspace}%
\newcommand{\uu}{u.\,U.\xspace}%
\newcommand{\ua}{u.\,a.\xspace}%
\newcommand{\idr}{i.\,d.\,R.\xspace}%
\newcommand{\vgl}{vgl.\xspace}%
\newcommand{\sa}{s.\,a.\xspace}%
\newcommand{\bzw}{bzw.\xspace}%
\newcommand{\evtl}{evtl.\xspace}%
\newcommand*{\eg}{e.\,g.\xspace}%
\newcommand*{\ie}{i.\,e.\xspace}%
% Leichtere Zitate
% ---------------------------------------
% \cite{} gut für Shorthand (Normen) im Text --> alternative zu \textcite
\newcommand{\autor}[2]{\textcite[#1]{#2}}%
\newcommand{\zitat}[2]{\parencite[#1]{#2}}%
\newcommand{\zitatpre}[3]{\parencite[#1][#2]{#3}}%
\newcommand{\zitate}[4]{\parencites[#1]{#2}[#3]{#4}}%
\newcommand{\zitatee}[6]{\parencites[#1]{#2}[#3]{#4}[#5]{#6}}%
\newcommand{\zitateee}[8]{\parencites[#1]{#2}[#3]{#4}[#5]{#6}[#7]{#8}}%
% ---------------------------------------
\newcommand{\insertref}{\todo[color=green!40]{Missing ref}}%
% Leichtere Bilder
% ---------------------------------------
\newcommand{\bild}[4]{%
	\begin{figure}[#1]
		\centering
		\includegraphics
		[
		width=#2\textwidth,
		]
		{figures/#3}
		
		\caption{#4}
		\label{fig:#3}
	\end{figure}
}
%
%
\newcommand{\bildsvg}[4]{
	\begin{figure}[#1]
		\centering%
		\def\svgwidth{#2\columnwidth}%
		\input{figures/#3}%
		%
		\caption{#4}%
		\label{fig:#3}%
	\end{figure}	
}
%
\newcommand{\bildtikz}[4]{
	\begin{figure}[#1]
		\centering
		\scalebox{#2}{\input{figures/#3}}
		\caption{#4}
		\label{fig:#3}
	\end{figure}
}
%
\newcommand{\bildtikzvar}[5]{
	\begin{figure}[#1]
		\centering
		\scalebox{#2}{\input{figures/#3}}
		\caption{\parbox[t]{#4\textwidth}{#5}}
		\label{fig:#3}
	\end{figure}
}
%
\newcommand{\bildtrim}[5]{% Mit definiertem Beschnitt
	\begin{figure}[#1]
		\centering
		\includegraphics
		[
		width=#2\textwidth,
		%keepaspectratio=true,
		trim=#3,		% l u r o
		clip,
		]
		{figures/#4}
		
		\caption[]{#5}
		\label{fig:#4}
	\end{figure}
}
%
\newcommand{\bildvar}[5]{% Mit definierter Breite / Zeilenumbruch der Caption
	\begin{figure}[#1]
		\centering
		\includegraphics
		[
		width=#2\textwidth,
		]
		{figures/#3}
		
		\caption[]{\parbox[t]{#4\textwidth}{#5}}
		\label{fig:#3}
	\end{figure}
}
%
\newcommand{\bildbox}[4]{% Bild mit Rahmen
	\begin{figure}[#1]
		\centering
		\fbox{%
			\includegraphics
			[
			width=#2\textwidth,
			]
			{figures/#3}
		}
		\caption[]{#4}
		\label{fig:#3}
	\end{figure}
}
%
\newcommand{\bildboxtrim}[5]{%
	\begin{figure}[#1]
		\centering
		\fbox{%
			\includegraphics
			[
			width=#2\textwidth,
			%keepaspectratio=true,
			trim=#3,		% l u r o
			clip,
			]
			{figures/#4}
		}
		\caption[]{#5}
		\label{fig:#4}
	\end{figure}
}
%
\newcommand{\bildboxvar}[5]{%
	\begin{figure}[#1]
		\centering
		\fbox{%
			\includegraphics
			[
			width=#2\textwidth,
			]
			{figures/#3}
		}
		\caption[]{\parbox[t]{#4\textwidth}{#5}}
		\label{fig:#3}
	\end{figure}
}
%
\newcommand{\bildsvgvar}[5]{
	\begin{figure}[#1]
		\centering
		\def\svgwidth{#2\columnwidth} 
		\subimport*{figures/}{#3}
		\caption{\parbox[t]{#4\textwidth}{#5}}
		\label{fig:#3}
	\end{figure}	
}
%
\newcommand{\bildpdf}[3]{%
	\begin{figure}[H]
		\centering
		\includegraphics
		[
		width=#1\textwidth,
		page={#2},
		]
		{#3}
		%\caption[]{#4}
		\label{fig:#3}
	\end{figure}
}
% ---------------------------------------