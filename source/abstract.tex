% In total max. 1 Page!
\IWBstudentthesisAbstract{%
	%
	% Abstract English:
	In the context of Industry 4.0, \gls{mas} plays a pivotal role in the 
	decentralization of \gls{cps}, and \gls{dt} contributes to the digitalization 
	of physical entities for \gls{plm}. This study aims to provide a solution for 
	\gls{mas} design in a more structured way, including the modularization of the 
	agent's capabilities based on Wannagat's architecture and the modularization of 
	network-based communication delays inherited from the soft and hard real-time 
	requirement of the automation and robot-like production system. The system 
	design consists of two parts: part 1, the local \gls{mas} to enable agent's 
	scheduling and coordination, decision-making and planning, and most 
	importantly, real-time communication under a 5G wireless network, and part 2, 
	the design of a \gls{dta} to transmit robot's process data to the global 
	\gls{dt} and back to the local device. Based on the system design, tests 
	have been conducted for both parts to analyze their timing properties under 
	different conditions, including worst-case scenarios. Additionally, two use 
	cases are applied to the local \gls{mas} system design, with the BMW use case 
	tested for its timing behaviors. Finally, a modular system design consideration 
	based on \gls{dsl} is provided, presented as a set of interconnected graphical notations 
	for the \gls{mas}. %
}{%
	%
	% Zusammenfassung Deutsch:
	Im Kontext von Industrie 4.0 spielt \gls{mas} eine zentrale Rolle bei der 
	Dezentralisierung von \gls{cps} und \gls{dt} trägt zur Digitalisierung 
	physischer Einheiten für \gls{plm} bei. Ziel dieser Studie ist es, eine 
	strukturiertere Lösung für das \gls{mas}-Design bereitzustellen, einschließlich 
	der Modularisierung der Fähigkeiten des Agenten auf der Grundlage der 
	Wannagat-Architektur und der Modularisierung netzwerkbasierter 
	Kommunikationsverzögerungen, die aus den weichen und harten 
	Echtzeitanforderungen übernommen wurden der Automatisierung und des 
	roboterähnlichen Produktionssystems. Das Systemdesign besteht aus zwei 
	Teilen: Teil 1, dem lokalen \gls{mas}, um die Scheduling und Koordination, 
	Entscheidungsfindung und Planung des Agenten und vor allem die 
	Echtzeitkommunikation unter einem 5G-Wireless-Netzwerk zu ermöglichen, 
	und Teil 2, der Entwurf eines \gls{dta} zur Übertragung der Prozessdaten 
	des Roboters an das globale \gls{dt} und zurück an das lokale Gerät. 
	Basierend auf dem Systemdesign wurden Tests für beide Teile durchgeführt, 
	um ihre Timing-Eigenschaften unter verschiedenen Bedingungen, einschließlich 
	Worst-Case-Szenarien, zu analysieren. Darüber hinaus werden zwei 
	Anwendungsfälle auf das lokale \gls{mas}-Systemdesign angewendet, 
	wobei der BMW-Anwendungsfall auf sein Zeitverhalten getestet wird. 
	Abschließend wird eine Überlegung zum modularen Systemdesign basierend 
	auf \gls{dsl} bereitgestellt, präsentiert als eine Reihe miteinander verbundener 
	grafischer Notationen für \gls{mas}.%
}%
%
%